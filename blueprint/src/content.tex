% In this file you should put the actual content of the blueprint.
% It will be used both by the web and the print version.
% It should *not* include the \begin{document}
%
% If you want to split the blueprint content into several files then
% the current file can be a simple sequence of \input. Otherwise It
% can start with a \section or \chapter for instance.
\chapter{Kjos-Hanssen (2010)}
Excerpts from the paper \emph{The probability distribution as a computational resource for randomness testing}.
\section{Introduction}

The fundamental idea of statistics is that by repeated experiment we can learn the underlying distribution of the phenomenon under investigation. In this paper we partially quantify the amount of randomness required to carry out this idea. We first show that ordinary Martin-L\"of randomness with respect to the distribution is sufficient. Somewhat surprisingly, however, the picture is more complicated when we consider a weaker form of randomness where the tests are \emph{effective}, rather than merely \emph{effective relative to the distribution}. We show that such \emph{Hippocratic} randomness actually coincides with ordinary randomness in that the same outcomes are random for each notion, but the corresponding test concepts do \emph{not} coincide: while there is a universal test for ordinary ML-randomness, there is none for Hippocratic ML-randomness.  

For concreteness we will focus on the classical Bernoulli experiment, although as the statistical tools we need are limited to Chebyshev's inequality and the strong law of large numbers, our result works also in  the general situation of repeated experiments in statistics, where an arbitrary sequence of independent and identically distributed random variables is studied. 

When using randomness as a computational resource, the most convenient underlying probability distribution may be that of a fair coin. In many cases, fairness of the proverbial coin may be only approximate. Imagine that an available resource generates randomness with respect to a distribution for which the probability of heads is $p\ne 1/2$. It is natural to assume that $p$ is not a computable number if the coin flips are generated with  contributions from a physical process such as the flipping of an actual coin. The non-computability of $p$ matters strongly if an infinite sequence of coin flips is to be performed. In that case, the gold standard of algorithmic randomness is \emph{Martin-L\"of randomness}, which essentially guarantees that no algorithm (using arbitrary resources of time and space) can detect any regularities in the sequence. If $p$ is non-computable, it is possible that $p$ may itself be a valuable resource, and so the question arises whether a ``truly random'' sequence should look random even to an adversary equipped with the distribution as a resource. %That is, should we require randomness that can fool only actual algorithms, or also algorithms that have access to an oracle for the real number $p$? We show that $p$ is essential for the existence of a universal randomness test, but not for any properties of random sequences themselves. While Martin-L\"of considered randomness for the case $p\ne 1/2$ in his original 1966 paper, he did not address this question. 
In this article we will show that the question is to some extent moot, as these types of randomness coincide. On the other hand, while there is a universal test for randomness in one case, in the other there is not.   This article can be seen as a follow-up to Martin-L\"of's paper where he introduced his notion of algorithmic randomness and proved results for Bernoulli measures \cite{ML}.%; see also Levin's paper \cite{Levin}.

It might seem that when testing for randomness, it is essential to have access to the distribution we are testing randomness for. On the other hand, perhaps if the results of the experiment are truly random we should be able to use them to discover the distribution for ourselves, and then once we know the distribution, test the results for randomness. However, if the original results are not really random, we may ``discover'' the wrong distribution.  We show that there are tests that can be effectively applied, such that if the results are random then the distribution can be discovered, and the results will then turn out to be random even to someone who knows the distribution. While these tests can individually be effectively applied, they cannot be effectively enumerated as a family. On the other hand, there is a single such test (due to Martin-L\"of) that will reveal whether the results are random for \emph{some} (Bernoulli) distribution, and another (introduced in this paper) that if so will reveal that distribution. 

In other words, one can effectively determine whether randomness for some distribution obtains, and if so determine that distribution. {There is no need to know the distribution ahead of time to test for randomness with respect to an unknown distribution.} %``Most'' sequences are not random for any iid experiment, but
If we suspect that a sequence is random with respect to a measure given by the value of a parameter (in an effective family of measures), there is no need to know the value of that parameter, as we can first use Martin-L\"of's idea to test for randomness with respect to \emph{some} value of the parameter, and then use the fundamental idea of statistics to \emph{find} that parameter.   Further effective tests can be applied to compare that parameter $q$ with rational numbers near our target parameter $p$, leading to the conclusion that {if all \emph{effective} tests for randomness with respect to parameter $p$ are passed, then all tests \emph{having access to $p$ as a resource} will also be passed}. But we need the distribution to know \emph{which} effective tests to apply. Thus we show that randomness testing with respect to a target distribution $p$ can be done by two agents each having limited knowledge: agent 1 has access to the distribution $p$, and agent 2 has access to the data $X$. Agent 1 tells agent 2 which tests to apply to $X$. %Thus, this task factors nontrivially, and the theoretician (who knows $p$) and the practitioner (who knows $X$) can be separate people. The theoretician can supply the tests to be applied without interacting with the practitioner, i.e. the tests to be applied do not depend on $X$. 

%EXPANDING ON THIS ANALOGY:
%The tests to be applied are a feature of the distribution itself. This can be contrasted with the situation in medicine, where various tests should be applied depending on how the patient responds. The point is not that medicine differs from randomness testing, but that the notion of a test in randomness is analogous to an entire procedure by which doctor and patient interacts, rather than the single ``tests'' administered by the doctor. Another point is that randomness testing is non-interactive: the tests do not alter the data, unlike how some medical tests can adversely affect the patient. The randomness test practitioner (``nurse'') does not need to know anything about the distribution that is being tested for (``medicine''). In other words the doctor can say: diagnosis is positive if any of the following tests fail (or perhaps succeed is a better word): test for this, test for this, test for this...in other words, the nurse can utilize a textbook or ``cookbook'' to find out what to do, without requiring any special insight into medicine. Thus randomness testing is like Julia Childs: if only the expert writes a good enough book, anyone can make delicious food (and Julia does not need to be updated on what they do).  Another analogy is with a football coach and a football player; in the randomness testing case, the coach can tell the player what to do, without taking into account what actually happens on game day.

The more specific point is that the information about the distribution $p$ required for randomness testing can be encoded in a set of effective randomness tests; and the encoding is intrinsic in the sense that the ordering of the tests does not matter, and further tests may be added: passing any collection of tests that include these is enough to guarantee randomness. From a syntactic point of view, whereas randomness with respect to $p$ is naturally a $\Sigma^0_2(p)$ class, our results show that it is actually an intersection of $\Sigma^0_{2}$ classes.

\begin{definition}\label{bernoulli}
	\lean{μ_bernoulli}
	\leanok
The Bernoulli measure $\mu_{p}$ is defined by the stipulation that for each $n\in\omega=\{0,1,2,\ldots\}$, 
\[
\mu_p(\{X: X(n)=1\})=p \text{ and }
\]
\[
\mu_p(\{X:X(n)=0\})=1-p
\]
and $X(0),X(1),X(2),\ldots$ are mutually independent random variables.
\end{definition}
If $X$ is a $\{0,1\}$-valued random variable such that $\mathbb P(X=1) = p$ then $X$ is called a Bernoulli($p$) random variable.
		

\begin{definition}\label{MLR}
	A $\mu_p$-ML-randomness test is a sequence $\{U^p_n\}_n$ that is uniformly $\Sigma^0_1(p)$ with $\mu(U^p_n)\le 2^{-n}$, where $2^{-n}$ may be replaced by any computable function that goes to zero effectively. 
	 
	A $\mu_p$-ML-randomness test is \emph{Hippocratic} if there is a $\Sigma^0_1$ class $S\subseteq 2^\omega\times\omega$ such that $S=\{(X,n):X\in U_n^p\}$. Thus, $U_n=U_n^p$ does not depend on $p$ and is uniformly $\Sigma^0_1$.  
If $X$ passes all $\mu_p$-randomness tests then $X$ is \emph{$\mu_p$-random}. If $X$ passes all Hippocratic tests then $X$ is \emph{Hippocrates $\mu$-random}.
\end{definition}

To explain the terminology: like the ancient medic Hippocrates we are not consulting the oracle of Delphi ($p$) but rather looking for ``natural causes''. This level of randomness recently arose in the study of randomness extraction from subsets of random sets \cite{MRL}.

We will often write ``$\mu_p$-random'' instead of ``$\mu_p$-ML-random'', as we work in the Martin-L\"of mode of randomness throughout, except when discussing a conjecture at the end of this paper.

%\newpage


\section{Chebyshev's inequality}

We develop this basic inequality from scratch here, in order to emphasize how generally it holds. For an event $A$ in a probability space, we let $\mathbf 1_A$, the indicator function of $A$, equal 1 if $A$ occurs, and $0$ otherwise. The \emph{expectation} of a discrete random variable $X$ is 
\[
\E(X)=\sum_x x\,\cdot\,\bbP(X=x).
\]
where $\bbP$ denotes probability and the sum is over all outcomes in the sample space. Thus $\E(X)$ is the average value of $X$ over repeated experiments. It is immediate that
\[
\E (\mathbf 1_{A}) = \bbP(A).
\]
Next we observe that the random variable that is equal to $a$ when a nonnegative random variable $X$ satisfies $X\ge a$ and $0$ otherwise, is always dominated by $X$. That is, 
\[
a\cdot \mathbf 1_{\{X\ge a\}} \le X.
\]
Therefore, taking expectations of both sides,
\[
a\cdot \bbP{\{X\ge a\}} \le \E(X).
\]
In particular, for any random variable $X$ with $\E(X)=\mu\in\mathbb R$ we have 
\[
a^2\cdot \bbP{\{(X-\mu)^2\ge a^2\}} \le \E ((X-\mu)^2)=:\sigma^2
\]
so
\[
\bbP{\{|X-\mu|\ge |a|\}} \le \sigma^2/a^2
\]
If we let $k\in\omega$ and replace $a$ by $k\sigma$, then
\[
\bbP{\{|X-\mu|\ge k\sigma\}} \le \sigma^2/(k\sigma)^2 = 1/k^2.
\]
This is Chebyshev's inequality, which in words says that the probability that we exceed the mean $\mu$ by $k$ many standard deviations $\sigma$ is rather small.

\chapter{Ahlman \& Koponen (2015)}
\begin{definition}
	\label{def:auto}
	\lean{automorphism_of_fin_k}
	\leanok
	Let $k\in\mathbb N$.
	An automorphisms of $\mathbb Z/ k\mathbb Z$ is a bijection $f$ that preserves addition:
	\[
	f(x+y)=f(x)+f(y)
	\]
\end{definition}
\begin{definition}
	\label{def:rigid}
	\lean{is_rigid}
	\leanok
	\uses{def:auto}
	The additive structure $\mathbb Z/ k\mathbb Z$ is rigid if it has no nontrivial automorphisms.
\end{definition}


\begin{theorem}
	\label{thm:rigid}
	\lean{fin2_rigid}
	\leanok
	%\notready
	\uses{def:auto}
	The additive structure $\mathbb Z/ 2\mathbb Z$ is rigid.
\end{theorem}
\begin{proof}
	Otherwise $f(0)=1$ and $f(1)=0$, but then
	\[
	0=f(1)=f(1+0)=f(1)+f(0)=0+1=1.
	\]
\end{proof}

\bibliographystyle{plain}
\bibliography{marginis}